\documentclass[letterpaper,12pt]{article}

% \usepackage{graphicx}
\usepackage[hmargin=3cm,vmargin=3.5cm]{geometry}
\usepackage{parskip}
\usepackage{fancyhdr}
\usepackage{microtype}

\title{Facilitator project report}
\author{Hans Chen}
\date{}

\pagestyle{fancy}

\fancyhead{}
\fancyhead[l]{Facilitator project report}
\fancyhead[R]{Hans Chen}


\begin{document}

\begin{center}
    \large Enhancing concise scientific writing and conceptual understanding in Earth system science
\end{center}

\section{Introduction}

I teach a Earth system science course for first-year Bachelor students in the Global Systems program.
This course focuses on developing students' ability to understand and describe Earth system processes from a systems perspective,
both qualitatively and quantitatively.

The main teaching and learning activities (TLAs) consist of lectures,
calculation exercises,
and assignments.
In the lectures,
students learn about different Earth system components,
the key processes within each component,
and interactions among components.
In the calculation exercises students work on quantitative problems related to lecture content.
Additionally,
students complete three assignments in which they run and develop simple Earth system models.
There are three dedicated help sessions to support students in completing the assignments.

The course assessment comprises two separate components:
the submitted assignments,
and a final  written exam at the end of the course.
The written exam consists of single-choice questions ($\sim$20\%),
written explanations ($\sim$50\%),
and calculations problems ($\sim$30\%).


\section{Challenge}

A significant portion of the assessment---%
about half of the final exam---%
is based on written answers,
yet there is a lack of TLAs that specifically target this skill.
We have noticed that some students struggle with providing short and direct answers to the questions.
Perhaps as a leftover from high school,
sometimes students write an overly long answers to hopefully cover some key words and ``fish'' for points.

To mitigate this,
we try to emphasize during both lectures and in the final exam that we are looking for short, concise answers to the questions,
and that we do not only look for key words,
but also grade based on the clarity of the answer.
We often also give an indication of the length we are looking for.
This is also reflected in the shared sample answers to old exam questions,
which are rarely more than a few sentences long.

Furthermore,
we have a lecture where students practice exam-type questions,
and we demonstrate how we would formulate an answer.
There is also an in-class activity during one of the lectures where students practice explaining different concepts to each other.
Nevertheless,
these are all in oral form,
and the students never get hands-on practice and feedback on written explanations.


\section{Aim}

The aim of this facilitator project is to introduce a set of TLAs that specifically target developing students' writing skills.
The aim is to prepare students to formulate direct, concise answers to scientific questions.

Through these TLAs,
I also hope to clarify to the students our expectations of the written answers on the final exam,
and make them understand how the answers are graded.
Another objective of these TLAs is to not only train students in wrting,
but also use the writing activities to develop their mastery and understanding of the course content.


\section{Proposed solution}

The proposed solution consists of four sequential TLAs.
As for the intended learning outcomes (ILOs),
the student should,
after the TLAs,
be able to:
(1) Demonstrate scientific understanding of the course material through short written answers, and
(2) Explain scientific concepts in a concise way without irrelevant information.


\subsection{TLA1: Rate sample answers}

Before class,
students will be provided sample questions and example answers of varying quality,
accompanied by the grading rubric used for assessment,
on the online learning platform Canvas.
The task is to evaluate and grade the sample answers.
After submitting their grading,
they will see how the teacher would have graded things,
along with written explanations of why certain grading decisions were made.

The aim of this TLA is to clarify the grading criteria,
provide examples of both good and bad answers,
and expose students to a different perspective: the teacher's perspective.
As this TLA can be carried out asynchronously,
it will be straightforward to coordinate.


\subsection{TLA2: In-class demonstration}

During class,
the teacher will show statistics of the graded exam questions from TLA1,
and explain their own grading for questions with varying grades.
Next,
the teacher will show a new question and show,
step by step,
how they would break it down and formulate an answer to it.

By showing the statistics of TLA1,
students will hopefully feel it was worthwhile to do the TLA.
Depending on the outcome,
it might also demonstrate the difficulty of grading and the lack of clarity of what is considered a good answer.

The objective of the in-class demonstration is to explicitly show an example of working systematically to forumlate an aswer.
Students have likely picked up a notation of how to do it during their school years and formed habits,
but some of these might not be good and need to be broken and retaught.
During this demonstration it is important to emphasize that this is just one approach,
and that the ``best'' approach might be individual.


\subsection{TLA3: Write and peer-review}

Following TLA2,
students in groups of 3--4 are provided sample questions and asked to formulate their own answers.
As this TLA focuses on the writing process rather than testing knowledge,
they are allowed to look at lecture materials,
use the course book,
etc.
However,
they will not be allowed to use AI-based tools,
to ensure they develop the writing skills themselves.

After formulating the answers,
students are provided the grading rubrics,
and are asked to assess the answers within the groups.
Based on all answers,
they are also asked to come up with an agreed-upon ``best'' answer.

This TLA will expose students to different approaches to answering questions.
By comparing and contrasting answers,
they will hopefully be able to identify what are good approcahes,
and what are less good.

The first attempts at answers,
group evaluations,
and final answer with teacher evaluation will be shared on Canvas after the class sessio
so that other groups can benefit from other groups' work.
Everything will be anonymous,
but there will also be an option to opt out of sharing personal answers,
but then you will not be able to access others' answers either.


\subsection{TLA4: Personal reflection}

Finally,
students will be asked to write a personal reflection on this acvity.
What was the most valuable thing they learned?
Did they find anything surprising?
Did they notice any common mistakes in their group's answers,
and how did they fix them?
Did they find the activities worthwhile?

Through this reflection,
the hope is that the students will recall what they have learned,
and also develop metacognition.
As part of the reflection will also be an evalation of the TLAs,
which will be part of the evaluation strategy.


\subsection{How it fits within the course}

These TLAs will fit the best when the students have relatively high mastery of the course material.
Thus,
the plan is to place them toward the end of the course.
This has the added benefit of motivating the students as a way to practice for the final exam.
Additionally,
this will introduce a delay between the lastly introduced material and the exam,
which will give students more time to digest the material and ask for support before the exam if needed.

This year we have decided to reduce the material in the course a bit and cut one 2-hour lecture block.
These TLAs could them nicely fill in this new empty block.


\section{Evaluation strategy}

Evaluating the effectiveness of these TLAs in teaching students how to concisely answer scientific questions will not be straightforward,
as the final exam also tests content mastery.
However,
one type of metric could be how often students ask for clarification on how questions are graded on the exam.
Currently,
there are often one or two students who wonder if ``wild guesses'' can affect the score negatively.
By clarifying what we mean by irrelevant information can lead to lower scores,
it will hopefully be clear that we are still fine with wild guesses;
however,
if one provides the right answer but adds irrelevant information on top,
it can affect the grading negatively.

One could try and see if the score on written answers improve after implementing the TLAs,
not only in this course,
but also the course that runs in parallel,
and perhaps also in later courses in the program.
Here I expect large variation from year to year and as mentioned also the effect of content mastery,
so it is not certain that the differences will be significant.

Another evaluation metric would be the answers in the submitted self reflections from TLA4.
This could be both quantitative (for example, grading on a scale from 1--5)
and qualitative (written answer).


\section{Discussion}

The assessment of these TLAs should be formulative and focus on the learning process.
The students will have to complete TLA1,
submit their group responses in TLA3,
and submit a personal reflection in TLA4 to get credit.
There will only be ``pass'' if everything is submitted,
or ``incomplete'' if something is missing.

I hesitate to make this a mandatory activity,
and would rather elect to provide bonus points on the exam upon completion.
The course ILOs and exam structure would have to be revisited to ensure that students meet all ILO when achieving a passing grade on the exam
when including the bonis points.
By keeping these TLAs lowstake,
the hope is to motivate students to participate without reservations.

I have been back and forth regarding how many times these TLAs should be given in the course.
One could argue that some of the time will be spent on understanding the TLAs themselves the first time,
thus it would be efficient to repeat them as students can focus more on the learning process the subsequent times.
However,
when considering the time allocation in the course,
it seems difficult to fit in multiple tries,
as well as motivate students to participate all times.
A potential solution,
if the evalation shows that this is effective,
could be to implement this on a program level,
where these (or similar) TLAs are carried out in for example two different courses across the program.

In summary,
this facilitator project has explored TLAs to teach students how to write short answers to scientific questions,
but also with the goal of help students learn the course material through the writing process.
The course is in study period 4 and has thus not been given again as of the time of writing,
but I look forward to test these ideas out in practice and evaluate their effectiveness to facilitate student learning and development.

\end{document}
