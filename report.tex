\documentclass[letterpaper,12pt]{article}

% \usepackage{graphicx}
\usepackage[hmargin=3cm,vmargin=3.5cm]{geometry}
\usepackage{parskip}
\usepackage{fancyhdr}
\usepackage{microtype}

\title{Facilitator project report}
\author{Hans Chen}
\date{}

\pagestyle{fancy}

\fancyhead{}
\fancyhead[l]{Facilitator project report}
\fancyhead[R]{Hans Chen}


\begin{document}

\begin{center}
    \large Enhancing concise scientific writing and conceptual understanding in Earth system science
\end{center}

\section{Introduction}

I teach a Earth system science course for first-year Bachelor students in the Global Systems program.
This course focuses on developing students' ability to understand and describe Earth system processes from a systems perspective,
both qualitatively and quantitatively.

The main teaching and learning activities (TLAs) consist of lectures,
calculation exercises,
and assignments.
In the lectures,
students learn about different Earth system components,
the key processes within each component,
and interactions among components.
In the calculation exercises students work on quantitative problems related to lecture content.
Additionally,
students complete three assignments in which they run and develop simple Earth system models.
There are three dedicated help sessions to support students in completing the assignments.

The course assessment comprises two separate components:
the submitted assignments,
and a final  written exam at the end of the course.
The written exam consists of single-choice questions ($\sim$20\%),
written explanations ($\sim$50\%),
and calculations problems ($\sim$30\%).


\section{Challenge}

A significant portion of the assessment---%
about half of the final exam---%
is based on written answers,
yet there is a lack of TLAs that specifically target this skill.
We have noticed that some students struggle with providing short and direct answers to the questions.
Perhaps as a leftover from high school,
sometimes students write an overly long answers to hopefully cover some key words and ``fish'' for points.

To mitigate this,
we try to emphasize during both lectures and in the final exam that we are looking for short, concise answers to the questions,
and that we do not only look for key words,
but also grade based on the clarity of the answer.
We often also give an indication of the length we are looking for.
This is also reflected in the shared sample answers to old exam questions,
which are rarely more than a few sentences long.

Furthermore,
we have a lecture where students practice exam-type questions,
and we demonstrate how we would formulate an answer.
There is also an in-class activity during one of the lectures where students practice explaining different concepts to each other.
Nevertheless,
these are all in oral form,
and the students never get hands-on practice and feedback on written explanations.


\subsection{Aim}

The aim of this facilitator project is to introduce a set of TLAs that specifically target developing students' writing skills.
The aim is to prepare students to formulate direct, concise answers to scientific questions.
% something about past research

Through these TLAs,
I also hope to clarify to the students our expectations of the written answers on the final exam,
and make them understand how the answers are graded.
Another objective of these TLAs is to not only train students in wrting,
but also use the writing activities to develop their mastery and understanding of the course content.


\section{Proposed solution}

The proposed solution consists of four sequential TLAs.
As for the intended learning outcomes (ILOs),
the student should,
after the TLAs,
be able to:
(1) Demonstrate scientific understanding of the course material through short written answers, and
(2) Explain scientific concepts in a concise way without irrelevant information.


\subsection{TLA1: Rate sample answers}

Before class,
students will be provided sample questions and example answers of varying quality,
accompanied by the grading rubric used for assessment,
on the online learning platform Canvas.
The task is to evaluate and grade the sample answers.
After submitting their grading,
they will see how the teacher would have graded things,
along with written explanations of why certain grading decisions were made.

The aim of this TLA is to clarify the grading criteria,
provide examples of both good and bad answers,
and expose students to a different perspective: the teacher's perspective.
As this TLA can be carried out asynchronously,
it will be straightforward to coordinate.


\subsection{TLA2: In-class demonstration}

During class,
the teacher will show statistics of the graded exam questions from TLA1,
and explain their own grading for questions with varying grades.
Next,
the teacher will show a new question and show,
step by step,
how they would break it down and formulate an answer to it.

By showing the statistics of TLA1,
students will hopefully feel it was worthwhile to do the TLA.
Depending on the outcome,
it might also demonstrate the difficulty of grading and the lack of clarity of what is considered a good answer.

The objective of the in-class demonstration is to explicitly show an example of working systematically to forumlate an aswer.
Students have likely picked up a notation of how to do it during their school years and formed habits,
but some of these might not be good and need to be broken and retaught.
During this demonstration it is important to emphasize that this is just one approach,
and that the ``best'' approach might be individual.


\subsection{TLA3: Write and peer-review}

Following TLA2,
students in groups of 3--4 are provided sample questions and asked to formulate their own answers.
As this TLA focuses on the writing process rather than testing knowledge,
they are allowed to look at lecture materials,
use the course book,
etc.
However,
they will not be allowed to use AI-based tools,
to ensure they develop the writing skills themselves.

After formulating the answers,
students are provided the grading rubrics,
and are asked to assess the answers within the groups.
Based on all answers,
they are also asked to come up with an agreed-upon ``best'' answer.

This TLA will expose students to different approaches to answering questions.
By comparing and contrasting answers,
they will hopefully be able to identify what are good approcahes,
and what are less good.

The first attempts at answers,
group evaluations,
and final answer with teacher evaluation will be shared on Canvas after the class sessios
so that other groups can benefit from other groups' work.


\subsection{TLA4: Personal reflection}




\subsection{How it fits within the course}




\section{Evaluation strategy}




\section{Discussion}




\end{document}
