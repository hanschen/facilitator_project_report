\documentclass[letterpaper,12pt]{article}

% \usepackage{graphicx}
\usepackage[hmargin=3cm,vmargin=3.5cm]{geometry}
\usepackage{parskip}
\usepackage{fancyhdr}
\usepackage{microtype}

\title{Facilitator project report}
\author{Hans Chen}
\date{}

\pagestyle{fancy}

\fancyhead{}
\fancyhead[l]{Facilitator project report}
\fancyhead[R]{Hans Chen}


\begin{document}

\begin{center}
    \large Enhancing concise scientific writing and conceptual understanding in Earth system science
\end{center}

\section{Introduction}

I teach a Earth system science course for first-year Bachelor students in the Global Systems program.
This course focuses on developing students' ability to understand and describe Earth system processes from a systems perspective,
both qualitatively and quantitatively.

The main teaching and learning activities (TLAs) consist of lectures,
calculation exercises,
and assignments.
In the lectures,
students learn about different Earth system components,
the key processes within each component,
and interactions among components.
In the calculation exercises students work on quantitative problems related to lecture content.
Additionally,
students complete three assignments in which they run and develop simple Earth system models.
There are three dedicated help sessions to support students in completing the assignments.

The course assessment comprises two separate components:
the submitted assignments,
and a final  written exam at the end of the course.
The written exam consists of single-choice questions ($\sim$20\%),
written explanations ($\sim$50\%),
and calculations problems ($\sim$30\%).


\section{Challenge}

A significant portion of the assessment---%
about half of the final exam---%
is based on written answers,
yet there is a lack of TLAs that specifically target this skill.
We have noticed that some students struggle with providing short and direct answers to the questions.
Perhaps as a leftover from high school,
sometimes students write an overly long answers to hopefully cover some key words and ``fish'' for points.

To mitigate this,
we try to emphasize during both lectures and in the final exam that we are looking for short, concise answers to the questions.
We often also give an indication of the length we are looking for.
This is also reflected in the shared sample answers to old exam questions,
which are rarely more than a few sentences long.

Furthermore,
we have a lecture where 

We emphasize in some lectures that we are looking for short, concise answers to the questions,
provide examples in one of the ``repetition'' lectures,
and also have an active learning exercise where students explain different concepts to each other.
These are however all in oral form,
and there is no active learning activity th



\end{document}
